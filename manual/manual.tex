\documentclass[a4paper]{article}
\usepackage[utf8]{inputenc}
\usepackage{fullpage}
\frenchspacing
%\pagestyle{empty}
\usepackage[pdftex,bookmarks=false,colorlinks,unicode=true]{hyperref}

\def\mulan{\textsc{MuLan-String}}

\title{MuLan-String library manual}
\author{felix-leg}

\begin{document}

\maketitle
\vfill
    Copyright © 2020  felix-leg. \par
    Permission is granted to copy, distribute and/or modify this document
    under the terms of the GNU Free Documentation License, Version 1.3
    or any later version published by the Free Software Foundation;
    with no Invariant Sections, no Front-Cover Texts, and no Back-Cover Texts.
    A copy of the license is included in the section entitled "GNU
    Free Documentation License".
\pagebreak
\tableofcontents

\pagebreak

\section{Introduction}

\mulan{} (or \textsc{MultiLanguage String}) is an internationalization and localization library made to aid translators whose native language nature forbids simple strings substitution eg. because it uses case system.
The other goal of \mulan{} is to make working with such languages simple for programmers writing programs with i18n in mind. 

The general idea of using \mulan{} is based on \emph{template strings}. The program using the library uses them for building strings from elements. 
The translators formulate template strings which can include rules of how to generate text based on language's case system, genders or pluralization rules.

This manual shows how to use \mulan{} from the view point of programmer and translator. One can read only the part made for one point of view and skip the other. 


\section{The programmer's POV}

\mulan{} was made to be easy to use in source code in different projects. Its main parts are the template system and the backend. 
Both parts are made with customization in mind. To use the library one have to "install" it in a project. 
To install the library you have to choose one of the files included with this manual:
\begin{description}
  \item[\texttt{mulanstring.hpp} file] contains only the core of the \mulan{} library. You can use this file, if you want to use your own backend.
  \item[\texttt{mulanstring-gettext.hpp} file] contains also the GNU Gettext backend bundled with the \mulan{}. If your project's internationalization system 
  is based on GNU Gettext library, you should choose this file
\end{description}

The installation itself is made around the idea of one file being both a header and an implementation\footnote{You may know it under the name "STB library"}.
For every source file in your project that requires \mulan{} you have just put a normal \verb+#include+ directive pointing to the file you have chosen from the above list.
\begin{verbatim}
#include "mulanstring.hpp"
\end{verbatim}
Then you have to choose \emph{one file} in your project that will contain the implementation like so:
\begin{verbatim}
#define MULAN_STRING_IMPLEMENTATION
#include "mulanstring.hpp"
\end{verbatim}
If you want to use \mulan{} with default options that's all you need. You are ready!

\subsection{Common options}
The previous section has shown the basic installation. Sometimes, however, you may want to change some tune up the library more to your needs.
\mulan{} offers you the bellow adjustments:
\begin{itemize}
  \item How to treat invalid templates,
  \item What are the template markers,
  \item (For GNU Gettext) if the library should include underscored functions
\end{itemize}

\paragraph{Invalid templates:} The first decision you may want is to choose how \mulan{} should react on improperly formatted template strings.
By default a wrong template object produces an empty string, no matter how it's called. If, for some reason, you want the \mulan{} to throw an error
you have to put a \verb+#define+ directive in the file that's including the library's implementation:
\begin{verbatim}
#define MULANSTR_THROW_ON_INVALID_TEMPLATE
#define MULAN_STRING_IMPLEMENTATION
#include "mulanstring.hpp"
\end{verbatim}

\paragraph{Template markers:}Another decision you may want to make is what delimiters are used in template strings. 
By default in \mulan{} templates are made using \verb+%{+ and \verb+}%+ markers. For example in:
\begin{quotation}
	\textbf{\%\{}\texttt{parent}\textbf{\}\%} has \textbf{\%\{}\texttt{num}\textbf{\}\%} kids
\end{quotation}
template the \verb+%{parent}%+ and \verb+%{num}%+ are substitution commands. 
If for some reason you want to use \verb+[[+ and \verb+]]+ as delimiters, you have to write:
\begin{verbatim}
#define MULANSTR_TAG_START "[["
#define MULANSTR_TAG_END "]]"
\end{verbatim} in the implementation file (before the \verb+#include+ directive).

Some template tags needs parameters. For example in:
\begin{quotation}
	There is \verb+%{num}%+ file\verb+%{num!P one=+\textbf{\{\}}\verb+ other=+\textbf{\{s\}}\verb+}%+
\end{quotation}
template the \textsc{P} function returns text based of the quantity given in the \texttt{num} variable. 
It then produces output based on whenever \texttt{num} is $1$ or $>$1 (for English language).
So, the above example will result in "There is 1 file" or "There is 8 files" (for \texttt{num} being $1$ and $8$ respectively).
For this it needs to know what text to output. And these texts are given between \textbf{\{} and \textbf{\}} delimiters.

If you want to change those internal delimiters to \verb+<+ and \verb+>+, you have to add:
\begin{verbatim}
#define MULANSTR_INNER_TAG_START "<"
#define MULANSTR_INNER_TAG_END ">"
\end{verbatim} to the above list.

\paragraph{Helper functions:}\label{helpFunc} The other option is to tell the library if we want to use 2 helper functions which all start with an underscore.
These are meant to speed up writing programs. They are short name replacements for functions retrieving template strings from the backend.
If you don't want them just write:
\begin{verbatim}
#define MULANSTR_DONT_USE_UNDERSCORE
\end{verbatim} in the known location.

These functions (and their replacements) are:
\begin{enumerate}
	\item \texttt{\_(message)} or \texttt{mls::translate(message)}: the basic template retriever. Gets template in the default language set during initialization.
	\item \texttt{\_c(catalog, message)} or \texttt{mls::translate(catalog, message)}: Gets template from a different catalog. 
	In the GNU Gettext `catalog' is the name of a `\texttt{.mo}` file.
\end{enumerate}


\subsection{Working with backends}
\mulan{}'s templates are taken from a backend. Different backends require different method to extract translatable strings from your project's code.
Another fact is some backends require an initialization process before they can retrieve strings. How to do it is explained below.

\subsubsection{GNU Gettext}
To initialize Gettext you need to call \verb+mls::backend::init(...)+ function. 
This function requires a name of the main catalog used by our program. Usually it is the project name set by \verb+#define PACKAGE ...+ in the \texttt{config.h} header (if you are using Autoconf).
Other parameters are: the locale name and localization of \texttt{.mo} files. These are optional and, if not provided, are set to their default values (which for the locale is the system locale 
and for the localization is \texttt{/usr/local/share/locale}). The initialization should be done as earlier as possible, preferably in the \verb+main()+ function:
\begin{verbatim}
int main() {
// ...
mls::backend::init(PACKAGE);
//or if you want to set locale:
mls::backend::init(PACKAGE, "en_US");
//or if you want to set localization:
mls::backend::init(PACKAGE, nullptr, "./locales");
//or all three:
mls::backend::init(PACKAGE, "en_US", "./locales");
// ...
}
\end{verbatim}

To extract template strings you have to run \texttt{xgettext} with these parameters: 
\begin{quote}
	\texttt{xgettext} \verb+-C+ \verb+-k_+ \verb+-k_c:2+ \texttt{-o} $<$\textbf{name of the output .pot file}$>$ $<$\textit{list of .hpp and .cpp files}$>$
\end{quote}
or, if you have set \verb+MULANSTR_DONT_USE_UNDERSCORE+ you need to write a longer list:
\begin{quote}
	\texttt{xgettext} \verb+-C+ \verb+-kmls::translate+ \verb+-kmls::translate:2+ $\backslash$ \\ \texttt{-o} $<$\textbf{name of the output .pot file}$>$ $<$\textit{list of .hpp and .cpp files}$>$
\end{quote}

For information on how to work with \texttt{.pot}, \texttt{.po} and \texttt{.mo} files refer to the GNU Gettext manual\footnote{Available at \url{https://www.gnu.org/software/gettext/manual/index.html}}.

\subsection{Using MLS templates in code}
The idea of \mulan{} is based around template strings. You use the library by making objects of the \verb+mls::Template+ class.
How to do it depends on if you preferred to use some backend or not:

\subsubsection{No backend}
This solution gives you the greatest freedom over how templates are constructed. It also may be the least comfortable, as you have to
provide also a \emph{locale} for every template object. To do so, call the \verb+mls::locale::getLocale(...)+ function:
\begin{verbatim}
auto myLocale = mls::locale::getLocale("en_US");
\end{verbatim}
(the locale names are listed in the \hyperref[supLangs]{Supported languages} section.)

To obtain a template object you have to make a \verb+mls::Template+ variable with \emph{template string} and \emph{locale} object passed to its constructor:
\begin{verbatim}
mls::Template aTemplate{"Template string", myLocale};
\end{verbatim}

\subsubsection{GNU Gettext backend}
Having GNU Gettext as our backend, you can get it by using special template retrieval functions discussed in the \ref{helpFunc} subsection. 
You can do it as in the example:
\begin{verbatim}
auto filesFound = _("%{num}% file%{num!P:,s}% have been found");
\end{verbatim}
This will give you template object with locale set during the initialization.

\subsubsection{Applying variables}
Some templates require additional information to produce a result string.
In the above example, the template needs \texttt{num} to be set for the number of files found.
We give that information by invoking \verb+apply(<var name>, <value>)+ method on the template:
\begin{verbatim}
filesFound.apply("num",3);
\end{verbatim}
Some templates have got more than one variable. To set them, we call \verb+apply(...)+ for each of the variables.
You can do it by separate calls or chain them:
\begin{verbatim}
auto parentHasKids = _("%{parent}% has got %{num}% %{num!P:kid,kids}%");

//1st method
parentHasKids.apply("parent", "Alice");
parentHasKids.apply("num", 3);
//2nd method
parentHasKids.apply("parent", "Alice").apply("num", 3);
\end{verbatim}

\subsubsection{Producing output}
After getting and applying variables (if necessary) we can get the result by calling \verb+get()+ method on the template.
This method produces a \verb+std::string+ which can be used elsewhere in the program.
\begin{verbatim}
std::cout << parentHasKids.get() << std::endl;
\end{verbatim}

\subsubsection{All three in one line}
All that was said above can be written in one line:
\begin{verbatim}
std::cout << _("%{parent}% has got %{num}% %{num!P:kid,kids}%")
			.apply("parent", "Bob").apply("num", 2)
			.get() << std::endl; //produces "Bob has got 2 kids"
\end{verbatim}

\subsection{The \texttt{apply} family}
What can you put in the \verb+apply(...)+ method? A couple of things:
\begin{description}
  \item[Numbers] You can set a template's variable to a number by using the \verb+apply(varName, number)+ method for integers up to the \verb+long+ size,
  or \verb+applyReal(varName, real)+ for real numbers up to the \verb+double+~size. Numbers can be used for their raw value or as an input to some functions.
  \item[Strings] Raw strings are printed as-is, without any formatting done to them. They should be used for anything that must not be or can not be modified 
  by a translator. Use them sparingly as this blocks your translator from applying any of \mulan{} powerful functions!
  \item[Other templates] When you want to insert some text into another text, then this is the right way of doing it. 
  It let's your translator to use the full potential of the \mulan{} template system. As a rule of thumb you should choose this method, if you want
  to put single words inside a sentence. These words should be made in their own template objects and retrieved from your backend. 
  
  If, for some reason, you can't do that (because, for example, these words are generated on-the-fly or come from some external source), you should fallback to raw strings.
  It would be then a good thing if you let your future translators get to know if this is the case (for example by using \mulan{} comments, see the next section).
\end{description}

\subsection{A quick overview of the template string syntax}
So far now I told you how to get and work with templates, but nothing about \emph{what to put in them}. 
This is the job of this subsection. However, I won't show all the possibles of the \mulan{} library.
The reason is you as the programmer are using English in the source code and as the default language for communicating with a user. 
So, you only need to know those elements of \mulan{}'s template system which are enough to work in English. 
If you want to know more, read the \hyperref[transPOV]{Translators' POV section} of this manual. 

OK, so what \mulan{}'s templates are made of? 
The templates are strings with tags which may be identified by \verb+%{+ and \verb+}%+ delimiters\footnote{Remember, you can change the delimiters}.
Everything inside them tells the \mulan{} to do its "magic". 

There are two main types of tags: a substitution tag and a function tag:
\paragraph{Substitutions} are tags in the form \verb+%{+\textit{variable\_name}\verb+}%+. They work by putting a text (or a number) from a variable named the same as it is written between the delimiters.
The content of that variable are given by the \verb+apply(...)+ method(s) of a template. 

\paragraph{Functions} are tags in the form \verb+%{+\textit{variable\_name}\verb+!FUNCTION+ \textit{arguments...}\verb+}%+.
They work in the similar way to the above, but use one of \mulan{} functions to process the content of the \textit{variable\_name} in some way and produce the output based on its result.
For the English users the only functions you must know about are:
\begin{itemize}
 \item the \texttt{P} function, which stands for \textit{(P)luralize},
 \item the \texttt{I} function that prints and formats an \textit{(I)nteger},
 \item the \texttt{R} function that prints and formats a \textit{(R)eal number}.
\end{itemize}

\subsubsection{Pluralizer}
The function gets two parameters and a variable name. The parameters tells the function what to output if the variable is equal to $1$ or not.

The arguments may be given in two possible ways:
\begin{itemize}
	\item \textbf{As a table}: \verb+%{+\textit{variable\_name}\verb+!P:+\textit{singular form}\verb+,+\textit{plural form}\verb+}%+
	\item \textbf{As a hash}: \verb+%{+\textit{variable\_name}\verb+!P one={+\textit{singular form}\verb+} other={+\textit{plural form}\verb+}}%+\footnote{Remember you can change internal \{
	and \} characters to anything else in your setup}
\end{itemize}

\paragraph{Example:}
\begin{quote}
	\verb+%{num}%+ \verb+%{num!P:page,pages}%+ \verb+%{num!P one={has} other={have}}%+ been printed.
\end{quote}
Suppose we have set \textit{num} to be equal $1$. The first tag in the above template will put the string "\texttt{1}" in its place.
The second tag will check the \textit{num} and, because it is equal $1$, it will put the "\texttt{page}" string in its place.
The third tag works the same way, but its argument list was given in a more verbose form. The tag in this case will get the string given in the \texttt{one} parameter
and put the "\texttt{has}" string in its place.

As a result the output will be: \texttt{"1 page has been printed."}.

Now, let's suppose the \textit{num} is equal to $4$. 
The first tag will produce "\texttt{4}", the second: "\texttt{pages}" and the third: "\texttt{have}" which results in an output:
\texttt{"4 pages have been printed."}.

\vspace{2em}

Take note that the above example may be written also in this way:
\begin{quote}
	\verb+%{num}%+ page\verb+%{num!P:,s}%+ ha\verb+%{num!P one={s} other={ve}}%+ been printed.
\end{quote}
The produced result will be the same. It's up to you which form you think is more readable and maintainable. 

\subsubsection{Integer formatter}
The next function in your (English) toolset is \texttt{I}. 
It is a simple function that prints an integer (and only an integer). What makes it different from simple \verb+%{var}%+ substitution? 
Well, it allows you (and your translator) to let your number string follow the formatting rules of an language. 
In English language it boils down to add ``,'' characters in numbers bigger than a thousand. 

You use this function by giving it an argument in this way:
\begin{itemize}
  \item \verb+%{+\textit{variable\_name}\verb+!I=general}%+ or
  \item \verb+%{+\textit{variable\_name}\verb+!I=grouped}%+
\end{itemize}
The first version prints number more of less as-is without grouping. The second one adds grouping characters when necessary. 

\paragraph{Example:}
\begin{quote}
  Your big number is \verb+%{num!I=general}%+ or \verb+%{num!I=grouped}%+ (pretty printed).
\end{quote}
Now if you set the \textit{num} variable to a million it will produce the string:
\begin{quote}
  Your big number is 1000000 or 1,000,000 (pretty printed).
\end{quote}

What to use depends on your business requirements.

\begin{tabular}{|p{.9\textwidth}|}
\hline %\\
  \textbf{Side note: Are there any other formats besides "general" and "grouped"?} \\
  You may wonder why the \texttt{I} function requires an argument. 
  Why to bother if one can simply print numbers with \verb+%{num}%+ and occasionally use \verb+%{num!I}%+ for cases
  when it requires the number to be pretty formatted? 
  Well, one case that you might consider is printing \emph{money} amounts. 
  This is one moment when you might think it would be enough to just print a number with thousand's separators like anywhere else
  (adjusted to the specific language's separator character). \\
  Unfortunately this way of thinking will let your \emph{Swiss} users get angry at you. 
  In this country there are different formatting rules for printing \emph{regular} numbers and the mentioned \emph{money amounts} numbers. 
  Normal numbers are formatted in this way: \\ 
  \texttt{1~234~567,89}\\ 
  And money are printed like that: \\ 
  \texttt{1'234'567.89} \\
  so forcing users of your application to chose only between two number formats may be not as well thought idea as you may think.
  \\
\hline
\end{tabular}

\subsubsection{Real numbers formatter}
The \texttt{R} function is used in a similar way as the \texttt{I} function. 
It let's you print numbers with fractional parts in a way that is compliant with the rules of the destination language (like for example using the right fraction separator character).

You can use this function in one of the below ways:
\begin{itemize}
  \item \verb+%{num!R:+\textit{number format}\verb+}%+
  \item \verb+%{num!R:+\textit{number format},\textit{precision}\verb+}%+
  \item \verb+%{num!R format={+\textit{number format}\verb+}}%+
  \item \verb+%{num!R prec={+\textit{precision}\verb+}}%+
  \item \verb+%{num!R format={+\textit{number format}\verb+} prec={+\textit{precision}\verb+}}%+
\end{itemize}
As you can see you may pass the arguments in both \emph{short} and \emph{verbose} format and specify the required \emph{precision} as well.
If you don't write precision, the \mulan{} assumes you want to print a number with as many fractional digits as it is necessary.

\paragraph{Example:}
\begin{quote}
  The Earth's radius is equal to \verb+%{radius!R:grouped,3}%+ km.
\end{quote}
This will result in a string:
\begin{quote}
  The Earth's radius is equal to $6~356.752$ km.
\end{quote}

\subsubsection{Comments} The last thing to know about \mulan{} templates is you can put comments inside them.
They are made in this way:
\begin{quote}
	Text \verb+%{#+\textit{comment}\verb+#}%+ around.
\end{quote}
What \mulan{} does with them is it treats them as if they weren't there. So, the result of the above example will be:
\begin{verbatim*}
Text  around.
\end{verbatim*}(mind the double spaces between words!)

What comments are useful for? Well, sometimes to translate the sentence translators need some external information. 
However the process of extracting templates from a source code may remove that information. Let's suppose we want translators to translate "Sam is beautiful."
In some languages the word "beautiful" will be different, depending on whenever Sam is a man or a woman. But all that a translator sees is just that simple text.
It would be good, if we had some way to inform our translators about the Sam's gender, thus solving the ambiguity. 

And that's is why comments may be useful. You can write the above example as: 
\begin{quote}
  Sam\verb+%{#a woman#}%+ is beautiful.
\end{quote}

\section{The translators' POV}\label{transPOV}

As a translator the only thing you must learn is the \mulan{}'s template syntax. 
The syntax is based on a system of tags inserted inside template string. They are recognized by being surrounded by delimiters.
These delimiters are, by default, \verb+%{+ and \verb+}%+. So, for example the below sentence:
\begin{quote}
	You have selected \verb+%{num}%+ \verb+%{num!P:object,objects}%+ for deletion.
\end{quote}
contains two tags: \verb+%{num}%+ and \verb+%{num!P:object,objects}%+. 
The first tag is called \emph{the substitution tag} and the second: \emph{the function tag}.
Everything that is surrounded by these delimiters is subject to \mulan{} "magic."

\paragraph{Warning!} The programmer has the ability to change those delimiters to anything else! Make sure you consult the programmers' team in case you would suspect they have done it.

\subsection{Types of tags}
OK, so what types of tag content you may encounter/use?
Well, at first let me tell you about general types of tags:
\subsubsection{Substitution tags}
This type is the simplest of all and also the most restricted in how it may behave. 
They are recognized by one word put between delimiters:
\begin{quote}
	\verb+%{+\textit{variable}\verb+}%+
\end{quote}
All what it does is to put the \textit{variable} content (given by a programmer) in its place.
You, as the translator, must make sure that content will be put in the right place. You do it by moving such a tag into a place where it is the most logical for a sentence in your language.

So, if given a template:\begin{quote}
	There were \verb+%{num}%+ changes in the \verb+%{document_type}%+
\end{quote}
If your language for some reason requires that \textit{document\_type} should be to put before \textit{num}, you can switch those two tags like that:\begin{quote}
	In \verb+%{document_type}%+ there is \verb+%{num}%+ changes.
\end{quote}
 The order of them doesn't matter as long as all of them are in place.

\subsubsection{Function tags}
Ah, there is where \mulan{} gets its power! This tag type returns the result of a call to one of predefined functions.
These functions works by getting the variable name (or not) and some parameters and, based on that information, produces an output. 

A function can take one parameter, a table, or a hash (also called a map). Also some functions needs a variable name to work, others doesn't need it.
If that sounds complicated, don't worry: I explain all of it in parts.

The first thing is whenever the function needs a variable or not:\begin{description}
	\item[Requred:] functions of this type are written in this form: \begin{quote}
		\verb+%{+\textit{variable\_name}\verb+!FUNCTION_NAME+\textit{argument(s)}\verb+}%+
	\end{quote} (mind the \texttt{!} character that separates the variable name from function name)
	\item[Not needed:] this type of functions are written in this form:\begin{quote}
		\verb+%{+\textbf{+}\verb+FUNCTION_NAME+\textit{argument(s)}\verb+}%+
	\end{quote} (remember about the plus sign at the beginning of the function name)
\end{description}

The second thing is how a function gets its arguments. An argument list goes right after the function name, and can take one of these forms:
\begin{center}
	\begin{tabular}{|l|l|l|} \hline 
		\bf Form & \bf Description & \bf Example \\ \hline 
		\verb+=+\textit{value} & gives \emph{one} parameter & \verb+%{item!C=gen}%+ \\ \hline 
		\verb+:+\textit{first}\verb+,+\textit{second}\verb+,+\ldots & gives \emph{a table} & \verb+%{num!P:mouse,mice}%+ \\ \hline 
		\verb*+ par1=+\textit{value1}\verb*+ par2=+\textit{value2}\ldots & gives \emph{a map} & \verb+%{person!G m={his} f={hers}}%+ \\ \hline 
	\end{tabular}
\end{center}

I hope all of it will be clear once you read the subsection \hyperref[funcs]{about functions}.

\subsubsection{Comment tags}

And the last are tags which serve as an aid for translators given by programmers. Sometimes you may find such that tag:
\begin{quote}
	\verb+%{#+\textit{a commentary}\verb+#}%+
\end{quote}
In a translation you can do anything to that tag, even remove it completely. 

What is their purpose? Well, sometimes you may find a template which is hard to translate without some external information. For example a text: \begin{quote}
	A kid just ran away.
\end{quote} may be hard or even impossible to translate, if in your language "ran away" requires the knowledge about the kid's sex. You then may ask a programmer to clarify it by putting a comment like this:
\begin{quote}
	A \verb+%{#male#}%+kid just ran away.
\end{quote}

Another reason is some backends like GNU Gettext may remove duplicate templates which, for some cases, can make a problem for a translator. Let's imagine we got this template:
\begin{quote}
	Open
\end{quote}
and an information that it is used in both "File" and "Print" menus. For some languages the translation of "Open" may differ depending on if it relates to a file or a printer.
Unfortunately, your backend may treat the second "Open" as redundant, leaving you with a problem how to write a translation which fits both cases.

The solution again lies in the possibility for programmer to add comments. For example:
\begin{quote}
	\verb+%{#File menu#}%+Open\\
	\verb+%{#Print menu#}%+Open
\end{quote} will not be reduced to one item and gives a translator a way to translate them differently. 

\subsection{List of functions}\label{funcs}

OK, so after we have learned how to write function tags, now I can tell you what possibilities \mulan{} gives to you.
I divided the list into 4 parts by the topic. Each topic tells about problems you may encounter during a translation.

\def\funSep{\vspace{1em} \hrule \vspace{1em}}

\subsubsection{Gender}
Gender tells about a class a word belongs to. It is a peculiarity of each noun and a language, which have got this feature, may require other words to adapt to that other word. 

\mulan{} gives two functions to work with genders:\par \funSep

{\large (S)et (G)ender $\Rightarrow$ \verb-%{+SG=-\textit{gender\_to\_set}\verb+}%+ }\vspace{1em}

Each template has got a gender assigned to them. Normally it is an empty string, but using the \textsc{SG} function we can change that.

This function produces nothing in place where it was put, so alone it is not useful. But becomes one when used with the next function.

\funSep {\large (G)ender writer $\Rightarrow$ \verb-%{noun!G:-\textit{a list}\verb-}%- \textbf{or} \verb-%{noun!G -\textit{a map}\verb-}%- }\vspace{1em}

Produces output based of gender set to the \textit{noun}. To work right the value of \textit{noun} variable must be another template where \textsc{SG} function was used.
\paragraph{Example:} Let's say we have a template (written in Latin):
\begin{quote}
	\verb+%{noun}%+ magn\verb+%{noun!G m={us} f={a} n={um}}%+ est
\end{quote}
and three other templates:
\begin{quote}
	\verb-%{+SG=m}%-Puteus\\
	\verb-%{+SG=f}%-Officina\\
	\verb-%{+SG=n}%-Forum
\end{quote}
If now a program sets the \textit{noun} variable to one of those three templates, the results will be:
\begin{quote}
	\texttt{Puteus magnus est} (for \textit{noun} set to the first template)\\
	\texttt{Officina magna est} (for the second template)\\
	\texttt{Forum magnum est} (and for the third template)
\end{quote}

\paragraph{Short and long version} In the example above the \textsc{G} function was written in the long (verbose) form. 
When executed it tries to match each of the given parameters names (\texttt{m}, \texttt{f} and \texttt{n}) to those set by the \textsc{SG} function in the sub-template. 
Sometimes however you don't want to write it \emph{that} long and want some shorter and quicker form to write. As you can see in the function header in this manual, 
the \textsc{G} function has a shorter form: \verb-%{noun!G:-\textit{a list}\verb-}%-. To use it in our Latin example we need to replace it with:
\begin{quote}
	\verb+%{noun}%+ magn\verb+%{noun!G:us,a,um}%+ est
\end{quote}
Now the \textit{-us}, \textit{-a} and \textit{-um} endings are assigned to the \texttt{m=}, \texttt{f=} and \texttt{n=} parameters in order.
The order of that parameters depends on the language. The \mulan{} has an embedded list of languages it supports. Each entry in that list contains, among others, 
a specific order set to the Gender writer function, which dictates how a list is converted into a map. You can read what is that order in the \hyperref[supLangs]{Supported languages section}.

\subsubsection{Case}

Some languages employ a case system. A form of a noun depends of what a function it has in a sentence. If, for example, a noun is used as sentence's subject it has different form than
when used as an object.

For this feature \mulan{} has got two functions:\par 

\funSep {\large (C)ase chooser $\Rightarrow$ \verb-%{noun!C=-\textit{required\_case\_name}\verb+}%+ }\vspace{1em}

Outputs the \textit{noun} variable, first informing it it should inflect by \textit{required\_case\_name}. 

\funSep {\large (C)ase writer $\Rightarrow$ \verb-%{+C case1={-\textit{case1\_output}\verb+} case2={+\textit{case2\_output}\verb+}}%+ \textbf{or}\\
\verb-%{+C:-\textit{list of outputs}\verb+}%+
 }\vspace{1em}

If a template containing this function was informed by the Case chooser to inflect, this function outputs a \texttt{case$n$\_output} associated with \texttt{case$n$}.

\paragraph{Example:} Again a Latin example. There is a template for a word \emph{domus} (house):
\begin{quote}
	dom\verb-%{+C nom={us} gen={us} dat={ui} acc={um} abl={o} voc={us} loc={i}}%-
\end{quote}
and a template:
\begin{quote}
	De \verb+%{from!C=abl}%+ in \verb+%{to!C=acc}%+
\end{quote} which means "From \textit{from} to \textit{to}".

If now the program assigns both variables to the same "domus" template, the output will be:
\begin{quote}
	De \textbf{domo} in \textbf{domum}
\end{quote} correctly inflected.

\paragraph{Short and long version} just like in the Gender section, the Case writer also has got a~long and a~short form.
The short form of the "domus" template may be like:
\begin{quote}
	dom\verb-%{+C:us,us,ui,um,o,us,i}%-
\end{quote}

And just like for Gender functions, the order of assigning elements of that list to a map is given in the \hyperref[supLangs]{Supported languages section}.

\subsubsection{Plural forms}
Pluralization rules tells how a noun should change its form based of some numerical quantity. Languages vary \emph{a lot} in this matter. 

There is only one function for this feature in \mulan{}.

\funSep {\large (P)luralizer $\Rightarrow$ \verb-%{num!P one={-\textit{output\_for\_1}\verb+} other={+\textit{output\_for\_the\_rest}\verb+}}%+ \textbf{or}\\
\verb-%{num!P:-\textit{list of outputs}\verb+}%+
 }\vspace{1em}
 
This function gets a number from the variable and gives an output depending of the value and pluralization rules in the template's language.

The \hyperref[supLangs]{Supported languages section} contains the rules, list of classes of numbers and how short form of the function is mapped to the long form.

\paragraph{Example:} In the English language nouns has got singular and plural forms. Singular forms are applied to quantities equal to $1$ and plural forms for the rest.
Therefore one can write a template like:
\begin{quote}
	\verb+%{num}%+ file\verb+%{num!P one={} other={s}}%+ deleted
\end{quote}
Now if a program sets \textit{num} variable to $1$ the output will be:
\begin{quote}
	1 file deleted
\end{quote}
and with $num=3$ it will be:
\begin{quote}
	3 files deleted
\end{quote}

\subsubsection{Number formats}
Functions in this section operates on numbers. They are meant to print numbers with the right thounsands separator and fractional separator.
Also these functions take care of separating numbers in groups with the right amount of digits in each group.

There are two functions in this group:

\funSep {\large (I)nteger formater $\Rightarrow$ \verb+%{num!I=+\textit{number format name}\verb+}%+}\vspace{1em}

It prints an integer (a number without any fractional parts). It takes one required parameter: \emph{number format name}. 
To know what are the names, refer to the \hyperref[supLangs]{Supported languages section}. However, mostly you can put there one of those values:
\begin{description}
  \item[general] It (usualy) prints a number without any separators. 
  \item[grouped] Prints a number grouped and separated.
\end{description}

What to use depends on what you think should be acceptable in your language in a specific template case.
You are free to change the programmer's decision. 

\funSep {\large (R)eal number formater $\Rightarrow$ \verb+%{num!R:+\textit{a list}\verb+}%+ \textbf{or} \verb+%{num!R +\textit{a map}\verb+}%+ }\vspace{1em}

This one is slightly more complicated function. It prints a real number (a number with some fractional parts).
It takes as parameters a number format name (used like in the previous function) and optionaly a precision.

It is used in one of these ways:
\begin{itemize}
  \item \verb+%{num!R:+\textit{number format}\verb+}%+
  \item \verb+%{num!R:+\textit{number format},\textit{precision}\verb+}%+
  \item \verb+%{num!R format={+\textit{number format}\verb+}}%+
  \item \verb+%{num!R prec={+\textit{precision}\verb+}}%+
  \item \verb+%{num!R format={+\textit{number format}\verb+} prec={+\textit{precision}\verb+}}%+
\end{itemize}

\paragraph{Example:}
Let's consider the template:
\begin{quote}
  You owe to the bank \$\verb+%{amount!R format{grouped} prec={2}}%+.
\end{quote}
When printed in English it may result in:
\begin{quote}
  You owe to the bank \$1,234.56.
\end{quote}

However, if you translate the above template to some of a Swiss bank application, you may write it like so:
\begin{quote}
  Sie schulden der Bank \verb+%{amount!R format={money} prec={2}}%+ CHF.
\end{quote}
which may result in this string (after doing the right variable substitution):
\begin{quote}
  Sie schulden der Bank 1'234.56 CHF.
\end{quote}

Take note of one thing: the programmer have chosen a \emph{grouped} number format, because that's the right format used when printing an amount of money in English.
However you, a supposedly Swiss translator, have changed that choice to special \emph{money} number format, that formats the value as it should be formatted in the Swiss language with regard to money.

\subsection{Supported languages}\label{supLangs}

This section contains the list of languages the \mulan{} has a builtin support. 
Take note that, if a language is not on the list, it doesn't stop the library from retrieving templates from a backend.
However it means that template functions \emph{will} output wrong texts. 

\def\none{\textit{None}}
\def\langName#1{\subsubsection*{#1}\hrule\vspace{1em}}

\langName{British English}
\begin{description}
	\item[Locale name:] \texttt{en\_GB}
	\item[Cases list:] \none
	\item[Genders list:] \none
	\item[Plurals list:] one($=1$), other($\not=1$)
	\item[Pluralization rule:] $\#1$
\end{description}

\langName{American English}
\begin{description}
	\item[Locale name:] \texttt{en\_US}
	\item[Cases list:] \none
	\item[Genders list:] \none
	\item[Plurals list:] one($=1$), other($\not=1$)
	\item[Pluralization rule:] $\#1$
\end{description}

\langName{Polish}
\begin{description}
	\item[Locale name:] \texttt{pl\_PL}
	\item[Cases list:] nom, gen, dat, acc, ins, loc, voc
	\item[Genders list:] m, f, n
	\item[Plurals list:] one($=1$), few($ending=[2,3,4]$ except $ending=[12,13,14]$), other
	\item[Pluralization rule:] $\#9$
\end{description}


\pagebreak
\appendix
\input{licence}

\end{document}
