\section{The translators' POV}\label{transPOV}

As a translator the only thing you must to learn is the \mulan{}'s template syntax. 
The syntax is based on a system of tags inserted inside template string. They are recognized by being surrounded by delimiters.
These delimiters are, by default, \verb+%{+ and \verb+}%+. So, for example the below sentence:
\begin{quote}
	You have selected \verb+%{num}%+ \verb+%{num!P:object,objects}%+ for deletion.
\end{quote}
contains two tags: \verb+%{num}%+ and \verb+%{num!P:object,objects}%+. 
The first tag is called \emph{the substitution tag} and the second: \emph{the function tag}.
Everything that is surrounded by these delimiters is subject to \mulan{} "magic."

\paragraph{Warning!} The programmer has the ability to change those delimiters to anything else! Make sure you consult the programmers' team in case you would suspect they have done it.

\subsection{Types of tags}
OK, so what types of tags content you may encounter/use?
Well, at first let me tell you about general types of tags:
\subsubsection{Substitution tags}
That type is the simplest of all and also the most restricted in how it may behave. 
They are recognized by one word put between delimiters:
\begin{quote}
	\verb+%{+\textit{variable}\verb+}%+
\end{quote}
All that it does is put the \textit{variable} content (given by a programmer) in its place.
You, as the translator, must make sure that content will be put in the right place. You do it by moving such a tag into a place where it is the most logical for a sentence in your language.

So, if given a template:\begin{quote}
	There were \verb+%{num}%+ changes in the \verb+%{document_type}%+
\end{quote}
If your language for some reason requires that \textit{document\_type} should be to put before \textit{num}, you can switch those two tags like that:\begin{quote}
	In \verb+%{document_type}%+ there is \verb+%{num}%+ changes.
\end{quote}
 The order of them doesn't matter as long as all of them are in place.

\subsubsection{Function tags}
Ah, there is where \mulan{} gets its power! That tag type returns the result of call to one of predefined functions.
These functions works by getting the variable name (or not) and some parameters and, based on that information, produces an output. 

A function can take one parameter, a table, or a hash (also called a map). Also some functions needs a variable name to work, others doesn't need it.
If that sounds complicated, don't worry: I explain all of it in parts.

The first thing is whenever the function needs a variable or not:\begin{description}
	\item[Requred:] functions of this type are written in this form: \begin{quote}
		\verb+%{+\textit{variable\_name}\verb+!FUNCTION_NAME+\textit{argument(s)}\verb+}%+
	\end{quote} (mind the \texttt{!} character that separates the variable name from function name)
	\item[Not needed:] this type of functions are written in this form:\begin{quote}
		\verb+%{+\textbf{+}\verb+FUNCTION_NAME+\textit{argument(s)}\verb+}%+
	\end{quote} (remember about the plus sign at the beginning of the function name)
\end{description}

The second thing is how a function gets its arguments. An argument list goes right after the function name, and can take one of these forms:
\begin{center}
	\begin{tabular}{|l|l|l|} \hline 
		\bf Form & \bf Description & \bf Example \\ \hline 
		\verb+=+\textit{value} & gives \emph{one} parameter & \verb+%{item!C=gen}%+ \\ \hline 
		\verb+:+\textit{first}\verb+,+\textit{second}\verb+,+\ldots & gives \emph{a table} & \verb+%{num!P:mouse,mice}%+ \\ \hline 
		\verb*+ par1=+\textit{value1}\verb*+ par2=+\textit{value2}\ldots & gives \emph{a map} & \verb+%{person!G m={his} f={hers}}%+ \\ \hline 
	\end{tabular}
\end{center}

I hope all of it will be clear once you read the subsection \hyperref[funcs]{about functions}.

\subsubsection{Comment tags}

And the last are tags which serve as an aid for translators given by programmers. Sometimes you may find such that tag:
\begin{quote}
	\verb+%{#+\textit{a commentary}\verb+#}%+
\end{quote}
In a translation you can do anything to that tag, even remove it completely. 

What is their purpose? Well, sometimes you may find a template which is hard to translate without some external information. For example a text: \begin{quote}
	A kid just ran away.
\end{quote} may be hard or even impossible to translate, if in your language "ran away" requires the knowledge about the kid's sex. You then may ask a programmer to clarify it by putting a comment like this:
\begin{quote}
	A \verb+%{#male#}%+kid just ran away.
\end{quote}

Another reason is some backends like GNU Gettext may remove duplicate templates which, for some cases, can make a problem for a translator. Let's imagine we got this template:
\begin{quote}
	Open
\end{quote}
and an information that it is used in both "File" and "Print" menus. For some languages the translation of "Open" may differ depending on if it relates to a file or a printer.
Unfortunately, your backend may treat the second "Open" as redundant, leaving you with a problem how to write a translation which fits both cases.

The solution again lies in the possibility for programmer to add comments. For example:
\begin{quote}
	\verb+%{#File menu#}%+Open\\
	\verb+%{#Print menu#}%+Open
\end{quote} will not be reduced to one item and gives a translator a way to translate them differently. 

\subsection{List of functions}\label{funcs}

OK, so after we learned how to write function tags, now I can tell you what possibilities \mulan{} gives to you.
I divided the list into 3 parts by the topic. Each topic tells about problems you may encounter during a translation.

\def\funSep{\vspace{1em} \hrule \vspace{1em}}

\subsubsection{Gender}
Gender tells about a class a word belongs to. it is a peculiarity of each noun and a language, which have got this feature, may require other words to adapt to the other word. 

\mulan{} gives two functions to work with genders:\par \funSep

{\large (S)et (G)ender $\Rightarrow$ \verb-%{+SG=-\textit{gender\_to\_set}\verb+}%+ }\vspace{1em}

Each template has got a gender assigned to them. Normally it is an empty string, but using the \textsc{SG} function we can change that.

This function produces nothing in place where it was put, so alone it is not useful. But becomes one when used with the next function.

\funSep {\large (G)ender writer $\Rightarrow$ \verb-%{noun!G:-\textit{a list}\verb-}%- \textbf{or} \verb-%{noun!G -\textit{a map}\verb-}%- }\vspace{1em}

Produces output based of gender set to the \textit{noun}. To work right the value of \textit{noun} variable must be another template where \textsc{SG} function was used.
\paragraph{Example:} Let's say we have a template (written in Latin):
\begin{quote}
	\verb+%{noun}%+ magn\verb+%{noun!G m={us} f={a} n={um}}%+ est
\end{quote}
and three other templates:
\begin{quote}
	\verb-%{+SG=m}%-Puteus\\
	\verb-%{+SG=f}%-Officina\\
	\verb-%{+SG=n}%-Forum
\end{quote}
If now a program sets \textit{noun} variable to one of those three templates, the results will be:
\begin{quote}
	\texttt{Puteus magnus est} (for \textit{noun} set to the first template)\\
	\texttt{Officina magna est} (for the second template)\\
	\texttt{Forum magnum est} (and for the third template)
\end{quote}

\paragraph{Short and long version} In the example above the \textsc{G} function was written in the long (verbose) form. 
When executed it tries to match each of the given parameters names (\texttt{m}, \texttt{f} and \texttt{n}) to those set by the \textsc{SG} function in the sub-template. 
Sometimes however you don't want to write it \emph{that} long and want some shorter and quicker form to write. As you can see in the function header in this manual, 
the \textsc{G} function has a shorter form: \verb-%{noun!G:-\textit{a list}\verb-}%-. To use it in our Latin example we need to replace it with:
\begin{quote}
	\verb+%{noun}%+ magn\verb+%{noun!G:us,a,um}%+ est
\end{quote}
Now the \textit{-us}, \textit{-a} and \textit{-um} endings are assigned to the \texttt{m=}, \texttt{f=} and \texttt{n=} parameters in order.
The order of that parameters depends on the language. The \mulan{} has an embedded list of languages it supports. Each of entry in that list contains, among others, 
a specific order set to the Gender writer function, which dictates how a list is converted into a map. You can read what is that order in the \hyperref[supLangs]{Supported languages section}.

\subsubsection{Case}

Some languages employ a case system. A form of a noun depends of what a function it has in a sentence. If for example a noun is used as sentence's subject it has different form than
when used as an object.

For this feature \mulan{} has got two functions:\par 

\funSep {\large (C)ase chooser $\Rightarrow$ \verb-%{noun!C=-\textit{required\_case\_name}\verb+}%+ }\vspace{1em}

Outputs the \textit{noun} variable, first informing it it should inflect by \textit{required\_case\_name}. 

\funSep {\large (C)ase writer $\Rightarrow$ \verb-%{+C case1={-\textit{case1\_output}\verb+} case2={+\textit{case2\_output}\verb+}}%+ \textbf{or}\\
\verb-%{+C:-\textit{list of outputs}\verb+}%+
 }\vspace{1em}

If a template containing this function was informed by the Case chooser to inflect, this function outputs a \texttt{case$n$\_output} associated with \texttt{case$n$}.

\paragraph{Example:} Again a Latin example. There is a template for word \emph{domus} (house):
\begin{quote}
	dom\verb-%{+C nom={us} gen={us} dat={ui} acc={um} abl={o} voc={us} loc={i}}%-
\end{quote}
and a template:
\begin{quote}
	De \verb+%{from!C=abl}%+ in \verb+%{to!C=acc}%+
\end{quote} which means "From \textit{from} to \textit{to}".

If now the program assigns both variables to the same "domus" template, the output will be:
\begin{quote}
	De \textbf{domo} in \textbf{domum}
\end{quote} correctly inflected.

\paragraph{Short and long version} just like in the Gender section, the Case writer also has got a~long and a~short form.
The short form of the "domus" template may be like:
\begin{quote}
	dom\verb-%{+C:us,us,ui,um,o,us,i}%-
\end{quote}

And just like for Gender functions, the order of assigning elements of that list to a map is given in the \hyperref[supLangs]{Supported languages section}.

\subsubsection{Plural forms}
Pluralization rules tells how a noun should change its form based of some numerical quantity. Languages vary \emph{a lot} in this matter. 

There is only one function for this feature in \mulan{}.

\funSep {\large (P)luralizer $\Rightarrow$ \verb-%{num!P one={-\textit{output\_for\_1}\verb+} other={+\textit{output\_for\_the\_rest}\verb+}}%+ \textbf{or}\\
\verb-%{num!P:-\textit{list of outputs}\verb+}%+
 }\vspace{1em}
 
The function gets a number from the variable and gives an output depending of the value and pluralization rules in the template's language.

The \hyperref[supLangs]{Supported languages section} contains the rules, list of classes of numbers and how short form of the function is mapped to the long form.

\paragraph{Example:} In the English language nouns has got singular and plural forms. Singular forms are applied to quantities equal $1$ and plural forms for the rest.
Therefore one can write a template like:
\begin{quote}
	\verb+%{num}%+ file\verb+%{num!P one={} other={s}}%+ deleted
\end{quote}
Now if a program sets \textit{num} variable to $1$ the output will be:
\begin{quote}
	1 file deleted
\end{quote}
and with $num=3$ it will be:
\begin{quote}
	3 files deleted
\end{quote}

