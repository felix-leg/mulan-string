\subsection{Supported languages}\label{supLangs}

This section contains the list of languages the \mulan{} has a builtin support. 
Take note that, if a language is not on the list, it doesn't stop the library from retrieving templates from a backend.
However it means that template functions \emph{will} output wrong texts. 

\def\none{\textit{None}}
\def\langName#1{\subsubsection*{#1}\hrule\vspace{1em}}

\langName{British English}
\begin{description}
	\item[Locale name:] \texttt{en\_GB}
	\item[Cases list:] \none
	\item[Genders list:] \none
	\item[Plurals list:] one($=1$), other($\not=1$)
	\item[Pluralization rule:] $\#1$
\end{description}

\langName{American English}
\begin{description}
	\item[Locale name:] \texttt{en\_US}
	\item[Cases list:] \none
	\item[Genders list:] \none
	\item[Plurals list:] one($=1$), other($\not=1$)
	\item[Pluralization rule:] $\#1$
\end{description}

\langName{Polish}
\begin{description}
	\item[Locale name:] \texttt{pl\_PL}
	\item[Cases list:] nom, gen, dat, acc, ins, loc, voc
	\item[Genders list:] m, f, n
	\item[Plurals list:] one($=1$), few($ending=[2,3,4]$ except $ending=[12,13,14]$), other
	\item[Pluralization rule:] $\#9$
\end{description}
