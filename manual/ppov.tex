\section{The programmer's POV}

\mulan{} was made to be easy to use in source code in different projects. Its main parts are the template system and the backend. 
Both parts are made with customization in mind. To use the library one have to "install" it in a project. 
To install the library you have to choose one of the files included with this manual:
\begin{description}
  \item[\texttt{mulanstring.hpp} file] contains only the core of the \mulan{} library. You can use this file, if you want to use your own backend.
  \item[\texttt{mulanstring-gettext.hpp} file] contains also the GNU Gettext backend bundled with the \mulan{}. If your project's internationalization system 
  is based on GNU Gettext library, you should choose this file
\end{description}

The installation itself is made around the idea of one file being both a header and an implementation\footnote{You may know it under the name "STB library"}.
For every source file in your project that requires \mulan{} you have just put a normal \verb+#include+ directive pointing to the file you have chosen from the above list.
\begin{verbatim}
#include "mulanstring.hpp"
\end{verbatim}
Then you have to choose \emph{one file} in your project that will contain the implementation like so:
\begin{verbatim}
#define MULAN_STRING_IMPLEMENTATION
#include "mulanstring.hpp"
\end{verbatim}
If you want to use \mulan{} with default options that's all you need. You are ready!

\subsection{Common options}
The previous section has shown the basic installation. Sometimes, however, you may want to change some tune up the library more to your needs.
\mulan{} offers you the bellow adjustments:
\begin{itemize}
  \item How to treat invalid templates,
  \item What are the template markers,
  \item (For GNU Gettext) if the library should include underscored functions
\end{itemize}

\paragraph{Invalid templates:} The first decision you may want is to choose how \mulan{} should react on improperly formatted template strings.
By default a wrong template object produces an empty string, no matter how it's called. If, for some reason, you want the \mulan{} to throw an error
you have to put a \verb+#define+ directive in the file that's including the library's implementation:
\begin{verbatim}
#define MULANSTR_THROW_ON_INVALID_TEMPLATE
#define MULAN_STRING_IMPLEMENTATION
#include "mulanstring.hpp"
\end{verbatim}

\paragraph{Template markers:}Another decision you may want to make is what delimiters are used in template strings. 
By default in \mulan{} templates are made using \verb+%{+ and \verb+}%+ markers. For example in:
\begin{quotation}
	\textbf{\%\{}\texttt{parent}\textbf{\}\%} has \textbf{\%\{}\texttt{num}\textbf{\}\%} kids
\end{quotation}
template the \verb+%{parent}%+ and \verb+%{num}%+ are substitution commands. 
If for some reason you want to use \verb+[[+ and \verb+]]+ as delimiters, you have to write:
\begin{verbatim}
#define MULANSTR_TAG_START "[["
#define MULANSTR_TAG_END "]]"
\end{verbatim} in the implementation file (before the \verb+#include+ directive).

Some template tags needs parameters. For example in:
\begin{quotation}
	There is \verb+%{num}%+ file\verb+%{num!P one=+\textbf{\{\}}\verb+ other=+\textbf{\{s\}}\verb+}%+
\end{quotation}
template the \textsc{P} function returns text based of the quantity given in the \texttt{num} variable. 
It then produces output based on whenever \texttt{num} is $1$ or $>$1 (for English language).
So, the above example will result in "There is 1 file" or "There is 8 files" (for \texttt{num} being $1$ and $8$ respectively).
For this it needs to know what text to output. And these texts are given between \textbf{\{} and \textbf{\}} delimiters.

If you want to change those internal delimiters to \verb+<+ and \verb+>+, you have to add:
\begin{verbatim}
#define MULANSTR_INNER_TAG_START "<"
#define MULANSTR_INNER_TAG_END ">"
\end{verbatim} to the above list.

\paragraph{Helper functions:}\label{helpFunc} The other option is to tell the library if we want to use 2 helper functions which all start with an underscore.
These are meant to speed up writing programs. They are short name replacements for functions retrieving template strings from the backend.
If you don't want them just write:
\begin{verbatim}
#define MULANSTR_DONT_USE_UNDERSCORE
\end{verbatim} in the known location.

These functions (and their replacements) are:
\begin{enumerate}
	\item \texttt{\_(message)} or \texttt{mls::translate(message)}: the basic template retriever. Gets template in the default language set during initialization.
	\item \texttt{\_c(catalog, message)} or \texttt{mls::translate(catalog, message)}: Gets template from a different catalog. 
	In the GNU Gettext `catalog' is the name of a `\texttt{.mo}` file.
\end{enumerate}


\subsection{Working with backends}
\mulan{}'s templates are taken from a backend. Different backends require different method to extract translatable strings from your project's code.
Another fact is some backends require an initialization process before they can retrieve strings. How to do it is explained below.

\subsubsection{GNU Gettext}
To initialize Gettext you need to call \verb+mls::backend::init(...)+ function. 
This function requires a name of the main catalog used by our program. Usually it is the project name set by \verb+#define PACKAGE ...+ in the \texttt{config.h} header (if you are using Autoconf).
Other parameters are: the locale name and localization of \texttt{.mo} files. These are optional and, if not provided, are set to their default values (which for the locale is the system locale 
and for the localization is \texttt{/usr/local/share/locale}). The initialization should be done as earlier as possible, preferably in the \verb+main()+ function:
\begin{verbatim}
int main() {
// ...
mls::backend::init(PACKAGE);
//or if you want to set locale:
mls::backend::init(PACKAGE, "en_US");
//or if you want to set localization:
mls::backend::init(PACKAGE, nullptr, "./locales");
//or all three:
mls::backend::init(PACKAGE, "en_US", "./locales");
// ...
}
\end{verbatim}

To extract template strings you have to run \texttt{xgettext} with these parameters: 
\begin{quote}
	\texttt{xgettext} \verb+-C+ \verb+-k_+ \verb+-k_c:2+ \texttt{-o} $<$\textbf{name of the output .pot file}$>$ $<$\textit{list of .hpp and .cpp files}$>$
\end{quote}
or, if you have set \verb+MULANSTR_DONT_USE_UNDERSCORE+ you need to write a longer list:
\begin{quote}
	\texttt{xgettext} \verb+-C+ \verb+-kmls::translate+ \verb+-kmls::translate:2+ $\backslash$ \\ \texttt{-o} $<$\textbf{name of the output .pot file}$>$ $<$\textit{list of .hpp and .cpp files}$>$
\end{quote}

For information on how to work with \texttt{.pot}, \texttt{.po} and \texttt{.mo} files refer to the GNU Gettext manual\footnote{Available at \url{https://www.gnu.org/software/gettext/manual/index.html}}.

\subsection{Using MLS templates in code}
The idea of \mulan{} is based around template strings. You use the library by making objects of the \verb+mls::Template+ class.
How to do it depends on if you preferred to use some backend or not:

\subsubsection{No backend}
This solution gives you the greatest freedom over how templates are constructed. It also may be the least comfortable, as you have to
provide also a \emph{locale} for every template object. To do so, call the \verb+mls::locale::getLocale(...)+ function:
\begin{verbatim}
auto myLocale = mls::locale::getLocale("en_US");
\end{verbatim}
(the locale names are listed in the \hyperref[supLangs]{Supported languages} section.)

To obtain a template object you have to make a \verb+mls::Template+ variable with \emph{template string} and \emph{locale} object passed to its constructor:
\begin{verbatim}
mls::Template aTemplate{"Template string", myLocale};
\end{verbatim}

\subsubsection{GNU Gettext backend}
Having GNU Gettext as our backend, you can get it by using special template retrieval functions discussed in the \ref{helpFunc} subsection. 
You can do it as in the example:
\begin{verbatim}
auto filesFound = _("%{num}% file%{num!P:,s}% have been found");
\end{verbatim}
This will give you template object with locale set during the initialization.

\subsubsection{Applying variables}
Some templates require additional information to produce a result string.
In the above example, the template needs \texttt{num} to be set for the number of files found.
We give that information by invoking \verb+apply(<var name>, <value>)+ method on the template:
\begin{verbatim}
filesFound.apply("num",3);
\end{verbatim}
Some templates have got more than one variable. To set them, we call \verb+apply(...)+ for each of the variables.
You can do it by separate calls or chain them:
\begin{verbatim}
auto parentHasKids = _("%{parent}% has got %{num}% %{num!P:kid,kids}%");

//1st method
parentHasKids.apply("parent", "Alice");
parentHasKids.apply("num", 3);
//2nd method
parentHasKids.apply("parent", "Alice").apply("num", 3);
\end{verbatim}

\subsubsection{Producing output}
After getting and applying variables (if necessary) we can get the result by calling \verb+get()+ method on the template.
This method produces a \verb+std::string+ which can be used elsewhere in the program.
\begin{verbatim}
std::cout << parentHasKids.get() << std::endl;
\end{verbatim}

\subsubsection{All three in one line}
All that was said above can be written in one line:
\begin{verbatim}
std::cout << _("%{parent}% has got %{num}% %{num!P:kid,kids}%")
			.apply("parent", "Bob").apply("num", 2)
			.get() << std::endl; //produces "Bob has got 2 kids"
\end{verbatim}

\subsection{The \texttt{apply} family}
What can you put in the \verb+apply(...)+ method? A couple of things:
\begin{description}
  \item[Numbers] You can set a template's variable to a number by using the \verb+apply(varName, number)+ method for integers up to the \verb+long+ size,
  or \verb+applyReal(varName, real)+ for real numbers up to the \verb+double+~size. Numbers can be used for their raw value or as an input to some functions.
  \item[Strings] Raw strings are printed as-is, without any formatting done to them. They should be used for anything that must not be or can not be modified 
  by a translator. Use them sparingly as this blocks your translator from applying any of \mulan{} powerful functions!
  \item[Other templates] When you want to insert some text into another text, then this is the right way of doing it. 
  It let's your translator to use the full potential of the \mulan{} template system. As a rule of thumb you should choose this method, if you want
  to put single words inside a sentence. These words should be made in their own template objects and retrieved from your backend. 
  
  If, for some reason, you can't do that (because, for example, these words are generated on-the-fly or come from some external source), you should fallback to raw strings.
  It would be then a good thing if you let your future translators get to know if this is the case (for example by using \mulan{} comments, see the next section).
\end{description}

\subsection{A quick overview of the template string syntax}
So far now I told you how to get and work with templates, but nothing about \emph{what to put in them}. 
This is the job of this subsection. However, I won't show all the possibles of the \mulan{} library.
The reason is you as the programmer are using English in the source code and as the default language for communicating with a user. 
So, you only need to know those elements of \mulan{}'s template system which are enough to work in English. 
If you want to know more, read the \hyperref[transPOV]{Translators' POV section} of this manual. 

OK, so what \mulan{}'s templates are made of? 
The templates are strings with tags which may be identified by \verb+%{+ and \verb+}%+ delimiters\footnote{Remember, you can change the delimiters}.
Everything inside them tells the \mulan{} to do its "magic". 

There are two main types of tags: a substitution tag and a function tag:
\paragraph{Substitutions} are tags in the form \verb+%{+\textit{variable\_name}\verb+}%+. They work by putting a text (or a number) from a variable named the same as it is written between the delimiters.
The content of that variable are given by the \verb+apply(...)+ method(s) of a template. 

\paragraph{Functions} are tags in the form \verb+%{+\textit{variable\_name}\verb+!FUNCTION+ \textit{arguments...}\verb+}%+.
They work in the similar way to the above, but use one of \mulan{} functions to process the content of the \textit{variable\_name} in some way and produce the output based on its result.
For the English users the only functions you must know about are:
\begin{itemize}
 \item the \texttt{P} function, which stands for \textit{(P)luralize},
 \item the \texttt{I} function that prints and formats an \textit{(I)nteger},
 \item the \texttt{R} function that prints and formats a \textit{(R)eal number}.
\end{itemize}

\subsubsection{Pluralizer}
The function gets two parameters and a variable name. The parameters tells the function what to output if the variable is equal to $1$ or not.

The arguments may be given in two possible ways:
\begin{itemize}
	\item \textbf{As a table}: \verb+%{+\textit{variable\_name}\verb+!P:+\textit{singular form}\verb+,+\textit{plural form}\verb+}%+
	\item \textbf{As a hash}: \verb+%{+\textit{variable\_name}\verb+!P one={+\textit{singular form}\verb+} other={+\textit{plural form}\verb+}}%+\footnote{Remember you can change internal \{
	and \} characters to anything else in your setup}
\end{itemize}

\paragraph{Example:}
\begin{quote}
	\verb+%{num}%+ \verb+%{num!P:page,pages}%+ \verb+%{num!P one={has} other={have}}%+ been printed.
\end{quote}
Suppose we have set \textit{num} to be equal $1$. The first tag in the above template will put the string "\texttt{1}" in its place.
The second tag will check the \textit{num} and, because it is equal $1$, it will put the "\texttt{page}" string in its place.
The third tag works the same way, but its argument list was given in a more verbose form. The tag in this case will get the string given in the \texttt{one} parameter
and put the "\texttt{has}" string in its place.

As a result the output will be: \texttt{"1 page has been printed."}.

Now, let's suppose the \textit{num} is equal to $4$. 
The first tag will produce "\texttt{4}", the second: "\texttt{pages}" and the third: "\texttt{have}" which results in an output:
\texttt{"4 pages have been printed."}.

\vspace{2em}

Take note that the above example may be written also in this way:
\begin{quote}
	\verb+%{num}%+ page\verb+%{num!P:,s}%+ ha\verb+%{num!P one={s} other={ve}}%+ been printed.
\end{quote}
The produced result will be the same. It's up to you which form you think is more readable and maintainable. 

\subsubsection{Integer formatter}
The next function in your (English) toolset is \texttt{I}. 
It is a simple function that prints an integer (and only an integer). What makes it different from simple \verb+%{var}%+ substitution? 
Well, it allows you (and your translator) to let your number string follow the formatting rules of an language. 
In English language it boils down to add ``,'' characters in numbers bigger than a thousand. 

You use this function by giving it an argument in this way:
\begin{itemize}
  \item \verb+%{+\textit{variable\_name}\verb+!I=general}%+ or
  \item \verb+%{+\textit{variable\_name}\verb+!I=grouped}%+
\end{itemize}
The first version prints number more of less as-is without grouping. The second one adds grouping characters when necessary. 

\paragraph{Example:}
\begin{quote}
  Your big number is \verb+%{num!I=general}%+ or \verb+%{num!I=grouped}%+ (pretty printed).
\end{quote}
Now if you set the \textit{num} variable to a million it will produce the string:
\begin{quote}
  Your big number is 1000000 or 1,000,000 (pretty printed).
\end{quote}

What to use depends on your business requirements.

\begin{tabular}{|p{.9\textwidth}|}
\hline %\\
  \textbf{Side note: Are there any other formats besides "general" and "grouped"?} \\
  You may wonder why the \texttt{I} function requires an argument. 
  Why to bother if one can simply print numbers with \verb+%{num}%+ and occasionally use \verb+%{num!I}%+ for cases
  when it requires the number to be pretty formatted? 
  Well, one case that you might consider is printing \emph{money} amounts. 
  This is one moment when you might think it would be enough to just print a number with thousand's separators like anywhere else
  (adjusted to the specific language's separator character). \\
  Unfortunately this way of thinking will let your \emph{Swiss} users get angry at you. 
  In this country there are different formatting rules for printing \emph{regular} numbers and the mentioned \emph{money amounts} numbers. 
  Normal numbers are formatted in this way: \\ 
  \texttt{1~234~567,89}\\ 
  And money are printed like that: \\ 
  \texttt{1'234'567.89} \\
  so forcing users of your application to chose only between two number formats may be not as well thought idea as you may think.
  \\
\hline
\end{tabular}

\subsubsection{Real numbers formatter}
The \texttt{R} function is used in a similar way as the \texttt{I} function. 
It let's you print numbers with fractional parts in a way that is compliant with the rules of the destination language (like for example using the right fraction separator character).

You can use this function in one of the below ways:
\begin{itemize}
  \item \verb+%{num!R:+\textit{number format}\verb+}%+
  \item \verb+%{num!R:+\textit{number format},\textit{precision}\verb+}%+
  \item \verb+%{num!R format={+\textit{number format}\verb+}}%+
  \item \verb+%{num!R prec={+\textit{precision}\verb+}}%+
  \item \verb+%{num!R format={+\textit{number format}\verb+} prec={+\textit{precision}\verb+}}%+
\end{itemize}
As you can see you may pass the arguments in both \emph{short} and \emph{verbose} format and specify the required \emph{precision} as well.
If you don't write precision, the \mulan{} assumes you want to print a number with as many fractional digits as it is necessary.

\paragraph{Example:}
\begin{quote}
  The Earth's radius is equal to \verb+%{radius!R:grouped,3}%+ km.
\end{quote}
This will result in a string:
\begin{quote}
  The Earth's radius is equal to $6~356.752$ km.
\end{quote}

\subsubsection{Comments} The last thing to know about \mulan{} templates is you can put comments inside them.
They are made in this way:
\begin{quote}
	Text \verb+%{#+\textit{comment}\verb+#}%+ around.
\end{quote}
What \mulan{} does with them is it treats them as if they weren't there. So, the result of the above example will be:
\begin{verbatim*}
Text  around.
\end{verbatim*}(mind the double spaces between words!)

What comments are useful for? Well, sometimes to translate the sentence translators need some external information. 
However the process of extracting templates from a source code may remove that information. Let's suppose we want translators to translate "Sam is beautiful."
In some languages the word "beautiful" will be different, depending on whenever Sam is a man or a woman. But all that a translator sees is just that simple text.
It would be good, if we had some way to inform our translators about the Sam's gender, thus solving the ambiguity. 

And that's is why comments may be useful. You can write the above example as: 
\begin{quote}
  Sam\verb+%{#a woman#}%+ is beautiful.
\end{quote}
