\section{The programmer's POV}

\mulan{} was made to be easy to use in source code in different projects. Its main parts are the template system and the backend. 
Both parts are made with customization in mind. To use the library one have to "install" it in a project. This is done by making 2 files:
\begin{description}
	\item[A *.hpp file] is meant to be included into every module which needs to be internationalized. 
	It contains some \verb+#define+-s and \verb+#include<mulanstring/main.hpp>+.
	\item[A *.cpp file] build once for the project. It also contains \verb+#define+-s and at the end \verb+#include+-s the \verb+mulanstring/main.cpp+ file. 
	The purpose of this file is to add the implementation part of the \mulan{} library into the project.
\end{description}

\subsection{The \texttt{mulanstr.hpp} file}\label{headerFile}
In this file we decide which of backends we want to use\footnote{for now (version $1.0$) \mulan{} offers only GNU Gettext backend}:
\begin{verbatim}
//to use GNU Gettext backend
#define MULANSTR_USE_GETTEXT
\end{verbatim}

The other option is to tell the library if we want to use 4 helper functions which all start with an underscore.
These are meant to speed up writing programs. They are short name replacements for functions retrieving template strings from the backend.
If one doesn't want them he has to write:
\begin{verbatim}
#define MULANSTR_DONT_USE_UNDERSCORE
\end{verbatim}

These functions (and their replacements) are:
\begin{enumerate}
	\item \texttt{\_(message)} or \texttt{mls::translate(message)}: the basic template retriever. Gets template in the default language set during initialization.
	\item \texttt{\_c(catalog, message)} or \texttt{mls::translate(catalog, message)}: Gets template from a different catalog. 
	The term `catalog' can mean different things for different backends. In the GNU Gettext it is the name of `\texttt{.mo}` file.
	\item \texttt{\_l(message, locale)} or \texttt{mls::translateWithLocale(message, locale)}: Gets template in the language given as the second parameter. 
	\item \texttt{\_cl(catalog, message, locale)} or\\ \texttt{mls::translateWithLocale(}\texttt{catalog, message, locale)}: like 2. but allows to choose a different language.
\end{enumerate}

And at the end of our header file we include the \mulan{} main header file:
\begin{verbatim}
#include <mulanstring/main.hpp>
\end{verbatim}

Now, in every source file in our project that needs to use \mulan{}'s elements, we include our file:
\begin{verbatim}
#include "mulanstr.hpp"
\end{verbatim}

\subsection{The \texttt{mulanstr.cpp} file}
This file properly installs \mulan{} into our project. 
Just like in the previous subsection, it does by \verb+#define+-ing macros, which affect the library's \verb+main.cpp+ file.

Again we first have to decide which backend to use:
\begin{verbatim}
//to use GNU Gettext backend
#define MULANSTR_USE_GETTEXT
\end{verbatim}

Then we can make a decision on what delimiters we use in template strings. 
By default in \mulan{} templates are made using \verb+%{+ and \verb+}%+. For example in:
\begin{quotation}
	\textbf{\%\{}\texttt{parent}\textbf{\}\%} has \textbf{\%\{}\texttt{num}\textbf{\}\%} kids
\end{quotation}
template the \verb+%{parent}%+ and \verb+%{num}%+ are substitution commands. 
If for some reason we want to use \verb+[[+ and \verb+]]+ as delimiters, we have to write:
\begin{verbatim}
#define MULANSTR_TAG_START "[["
#define MULANSTR_TAG_END "]]"
\end{verbatim} in our \texttt{mulanstr.cpp} file.

Some template inserts needs parameters. For example in:
\begin{quotation}
	There is \verb+%{num}%+ file\verb+%{num!P one=+\textbf{\{\}}\verb+ other=+\textbf{\{s\}}\verb+}%+
\end{quotation}
template the \textsc{P} function returns text based of the quantity given in \texttt{num} variable. 
It then produces output based on whenever \texttt{num} is $1$ or $>$1 (for English language).
So, the above example will result in "There is 1 file" or "There is 8 files" (for \texttt{num} being $1$ and $8$ respectively).
For this it needs to know what text to output. And these texts are given between \textbf{\{} and \textbf{\}} delimiters.

If one wants to change those internal delimiters to \verb+<+ and \verb+>+, he has to write:
\begin{verbatim}
#define MULANSTR_INNER_TAG_START "<"
#define MULANSTR_INNER_TAG_END ">"
\end{verbatim}

And finally our \texttt{mulanstr.cpp} file has to end with:
\begin{verbatim}
#include <mulanstring/main.cpp>
\end{verbatim}
\subsection{Using MLS templates in code}
OK, once we have set those two files properly, we can use \mulan{}'s capabilities. 
To do it we have to find in our project's source code every use of strings which needs to be translated. 
Then we have to replace these strings with \mulan{}'s call to templates. It consist of three parts:
\subsubsection{Getting the template}
First we need a template. We get it by using special template retrieval functions discussed in the \ref{headerFile} subsection. 
We can do it as in the example:
\begin{verbatim}
auto filesFound = _("%{num}% file%{num!P:,s}% have been found");
\end{verbatim}

\subsubsection{Applying variables}
Some templates require additional information to produce a result string.
In the above example, the template needs \texttt{num} to be set for the number of files found.
We give that information by invoking \verb+apply(<var name>, <value>)+ method on the template:
\begin{verbatim}
filesFound.apply("num",3);
\end{verbatim}
Some templates have got more than one variable. To set them we call \verb+apply(...)+ for each of the variables.
You can do it by separate calls or chain them:
\begin{verbatim}
auto parentHasKids = _("%{parent}% has got %{num}% %{num!P:kid,kids}%");

//1st method
parentHasKids.apply("parent", "Alice");
parentHasKids.apply("num", 3);
//2nd method
parentHasKids.apply("parent", "Alice").apply("num", 3);
\end{verbatim}

\subsubsection{Producing output}
After getting and applying variables (if necessary) we can get the result by calling \verb+get()+ method on the template.
This method produces a \verb+std::string+ which can be used elsewhere in the program.
\begin{verbatim}
std::cout << parentHasKids.get() << std::endl;
\end{verbatim}

\subsubsection{All three in one line}
All that was said above can be written in one line:
\begin{verbatim}
std::cout << _("%{parent}% has got %{num}% %{num!P:kid,kids}%")
			.apply("parent", "Bob").apply("num", 2)
			.get() << std::endl; //produces "Bob has got 2 kids"
\end{verbatim}

\subsection{Working with backends}
\mulan{}'s templates are taken from the backend. Different backends require different method to extract translatable strings from your project's code.
Another fact is some backends require an initialization process before they can retrieve strings. How to do it is explained below.
\subsubsection{GNU Gettext}
To initialize Gettext we need to call \verb+mls::backend::init(...)+ function. 
This function requires a name of the main catalog used by our program. Usually it is project name set by \verb+#define PACKAGE ...+ in the \texttt{config.h} header (if we use Autoconf).
Other parameters are: the locale name and localization of \texttt{.mo} files. These are optional and, if not provides, are set to the default values (which for the locale is the system locale 
and for the localization is \texttt{/usr/local/share/locale}). The initialization should be done as earlier as possible, preferably in the \verb+main()+ function:
\begin{verbatim}
int main() {
// ...
mls::backend::init(PACKAGE);
//or if we want to set locale:
mls::backend::init(PACKAGE, "en_US");
//or if we want to set localization:
mls::backend::init(PACKAGE, nullptr, "./locales");
//or all three:
mls::backend::init(PACKAGE, "en_US", "./locales");
// ...
}
\end{verbatim}

To extract template strings we need to run \texttt{xgettext} with these parameters: 
\begin{quote}
	\texttt{xgettext} \verb+-C+ \verb+-k_+ \verb+-k_c:2+ \verb+-k_l:1+ \verb+-k_cl:2+ \texttt{-o} $<$\textbf{name of .pot file}$>$ $<$\textit{list of .hpp and .cpp files}$>$
\end{quote}
or, if we set \verb+MULANSTR_DONT_USE_UNDERSCORE+ we need to write longer list:
\begin{quote}
	\texttt{xgettext} \verb+-C+ \verb+-kmls::translate+ \verb+-kmls::translate:2+ $\backslash$ \\ \verb+-kmls::translateWithLocale:1+ \verb+-kmls::translateWithLocale:2+ $\backslash$ \\ \texttt{-o} $<$\textbf{name of .pot file}$>$ $<$\textit{list of .hpp and .cpp files}$>$
\end{quote}

For information on how to work with \texttt{.pot}, \texttt{.po} and \texttt{.mo} files refer to the GNU Gettext manual\footnote{Available at \url{https://www.gnu.org/software/gettext/manual/index.html}}.

\subsection{A quick overview of the template string syntax}
So far now I showed how to get and work with templates, but nothing about \emph{what to put in them}. 
This is the job of this subsection. However, I won't show all the possibles of the \mulan{} library.
The reason is you as the programmer are using English in the source code and as the default language for communicating with a user. 
So, you only need to know those elements of \mulan{}'s template system which are enough to work in English. 
If you want to know more, read the \hyperref[transPOV]{Translators' POV section} of this manual. 

OK, so what \mulan{}'s templates are made of? 
The templates are strings with tags which may be identified by \verb+%{+ and \verb+}%+ delimiters\footnote{Remember, you can change the delimiters}.
Everything inside them tells the \mulan{} to do its magic. 

There are two main types of tags: a substitution tag and a function tag:
\paragraph{The substitutions} are tags in the form \verb+%{+\textit{variable\_name}\verb+}%+. They work by putting a text (or a number) from a variable named the same as it is written between the delimiters.
The content of that variable are given by the \verb+apply(...)+ method of the template. 

\paragraph{The functions} are tags in the form \verb+%{+\textit{variable\_name}\verb+!FUNCTION+ \textit{arguments...}\verb+}%+.
They work in the similar way to the above, but uses one of \mulan{} functions to process the content of the \textit{variable\_name} in some way and produce the output based on the result.
For the English users the only function you must know about is the \texttt{P} function, which stands for \textit{(P)luralize}.
The function gets two parameters and a variable name. The parameters tells the function what to output if the variable is equal to $1$ or not.

The arguments may be given in two possible ways:
\begin{itemize}
	\item \textbf{As a table}: \verb+%{+\textit{variable\_name}\verb+!P:+\textit{singular form}\verb+,+\textit{plural form}\verb+}%+
	\item \textbf{As a hash}: \verb+%{+\textit{variable\_name}\verb+!P one={+\textit{singular form}\verb+} other={+\textit{plural form}\verb+}}%+\footnote{Remember you can change internal \{
	and \} characters to anything else}
\end{itemize}

\paragraph{Example:}
\begin{quote}
	\verb+%{num}%+ \verb+%{num!P:page,pages}%+ \verb+%{num!P one={has} other={have}}%+ been printed.
\end{quote}
Suppose we set \textit{num} to be equal $1$. The first tag in the above template will put the string "\texttt{1}" in its place.
The second tag will check the \textit{num} and, because it is equal $1$, it will put the "\texttt{page}" string in its place.
The third tag works the same way, but its argument list was given in a more verbose form. The tag in this case will get the string given in the \texttt{one} parameter
and put the "\texttt{has}" string in its place.

As a result the output will be: \texttt{"1 page has been printed."}.

Now, let's suppose the \textit{num} is equal to $4$. 
The first tag will produce "\texttt{4}", the second: "\texttt{pages}" and the third: "\texttt{have}" which results in a output:
\texttt{"4 pages have been printed."}.

\vspace{2em}

Take note that the above example may be written also in this way:
\begin{quote}
	\verb+%{num}%+ page\verb+%{num!P:,s}%+ ha\verb+%{num!P one={s} other={ve}}%+ been printed.
\end{quote}
The produced result will be the same. It's up to you which form you think is more readable and maintainable. 

\paragraph{Comments} The last thing to know about \mulan{} templates is you can put comments inside them.
They are made in this way:
\begin{quote}
	Text \verb+%{#+\textit{comment}\verb+#}%+ around.
\end{quote}
What \mulan{} does with them is it treats them as if they wasn't there. So, the result of above example will be:
\begin{verbatim*}
Text  around.
\end{verbatim*}(mind the double spaces between words!)

What comments are useful for? Well, sometimes to translate the sentence translators need some external information. 
However the process of extracting templates from a source code removes that information. Let's suppose we want translators to translate "Sam is beautiful."
In some languages the word "beautiful" will be different, depending on whenever Sam is a man or a woman. But all that a translator sees is just that simple text.
It would be good, if we had some way to inform our translators about the Sam's gender, thus solving the ambiguity. 

And that's is why comments may be useful. You can write the above example as: "Sam\verb+%{#a woman#}%+ is beautiful."
